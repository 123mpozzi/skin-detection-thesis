%%%%%%%%%%%%%%%%%%%%%%%%%%%%%%%%%%%%%%%%%%%%%%%%%%%%%%%%%%%%%%%%%%%%%%%%
\chapter{Conclusion}\label{chap:conclusion}
%%%%%%%%%%%%%%%%%%%%%%%%%%%%%%%%%%%%%%%%%%%%%%%%%%%%%%%%%%%%%%%%%%%%%%%%


% cosa è stato fatto
In this thesis the significance and limitations of skin detection have been addressed.
A review of public datasets available in the domain and an analysis of state-of-the-art approaches has been presented, including a new proposed taxonomy.
Three different state-of-the-art methods have been thoroughly examined implemented and validated in respect to the original papers, when possible.
An evaluation of the chosen approaches in different settings has been presented, alongside a discussion on the metrics used in the domain.
Finally, the results have been thoroughly discussed through data and figures.


%%%%%%%%%%%%%%%%%%%%%%%%%%%%%%%%%%%%%%%%%%%%%%%%%%%%%%%%%%%%%%%%%%%%%%%%
\section{Discussion}
%%%%%%%%%%%%%%%%%%%%%%%%%%%%%%%%%%%%%%%%%%%%%%%%%%%%%%%%%%%%%%%%%%%%%%%%

% cosa è risultato
The analysis of the evaluation results has pointed out the strengths and weaknesses of each of the chosen approaches.\\
The thresholding method reported the worst classification scores but described inference times dozens of times faster on the CPU.
Moreover, its prediction time resulted independent of image pixels.
The prediction is inaccurate when the images have dark skin tones and backgrounds with skin-like colors.\\
The statistical approach reported better results, but was prone to several False Positives.
Being color-based, it also shared some limitations with the thresholding approach: materials with skin-like colors have represented a challenge to classify.\\
The U-Net approach has always reported the best scores.
It demonstrated how deep learning is able to find features other than color to be able to classify skin pixels even in tricky situations.
It struggled to generalize in two cases: when the training dataset was very small and when the training data was too different from the test data.\\
The involvement of more balanced metrics has proven to be essential to clarify situations where other metrics described over-optimistic results.


%%%%%%%%%%%%%%%%%%%%%%%%%%%%%%%%%%%%%%%%%%%%%%%%%%%%%%%%%%%%%%%%%%%%%%%%
\section{Future Work}
%%%%%%%%%%%%%%%%%%%%%%%%%%%%%%%%%%%%%%%%%%%%%%%%%%%%%%%%%%%%%%%%%%%%%%%%

Skin detection approaches have evolved in multiple paths, covering aspects such as classification performance and computational efficiency.
However, skin detection datasets still represent a major limitation in the development and evaluation of skin detectors.
New datasets are still highly sought after and could represent a significant boost in the skin detection domain.

In future evaluations, it should be taken into account the metric behavior and its limitations.
Metrics that describe a more stable behavior on unbalanced datasets can be involved to represent results in a more complete way.

Finally, the development of skin detectors could benefit from the progress that image segmentation is having with deep learning, especially in the medical field, which often features binary classification problems.\\
For achieving better classifications, Transformers could be considered, as they have proven to be really solid in Natural Language Processing~\cite{vaswani2017attention}, and are starting to gain traction in the image segmentation tasks, so much that some U-Net-like architectures have been recently designed~\cite{cao2021swin, chen2021transunet}.\\
Regarding performance and computational powers, skin detectors could venture into mobile deep learning development.
In fact, mobile phones, are starting to become a solid platform for U-Nets~\cite{ignatov2021fast}.

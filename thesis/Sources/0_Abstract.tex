\begin{center}
  \textsc{Abstract}
\end{center}
%
\noindent
% CONTEXT
Skin detection is the process of discriminating skin and non-skin pixels in an arbitrary image and represents an intermediate step in several image processing tasks, such as facial analysis and biomedical segmentation.
% PROBLEM to overcome
Different approaches have been presented in the literature, but a comparison is difficult to perform due to multiple datasets and varying performance measurements.
% HOW i addressed it
In this work, the datasets and the state-of-the-art approaches are reviewed and categorized using a new proposed taxonomy.
% WHAT i have done
Three different representative skin detector methods of the state of the art are selected and thoroughly analyzed. This approaches are then evaluated on three different state of the art datasets and skin tones sub-datasets using multiple metrics.
The evaluation is performed on single and cross dataset scenario to highlight key differences between methods, reporting also the inference time.
Finally, the results are organized into multiple tables, using the related figures as an assistance tool to support the discussion. %to exhibit meaningful instances.
% KEY IMPACT of the work
Experimental results demonstrate the strength and weaknesses of each approach, and the need to involve multiple metrics for a fair assessment of the method's aspects.
